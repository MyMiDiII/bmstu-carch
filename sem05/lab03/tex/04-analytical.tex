\chapter{Основные теоретические сведения}

В данном разделе представлено описание программы \textbf{PCLAB}, ипользуемой
при исследовании производительности в данной лабораторной работе. 

\section{Программа  PCLAB}

Программа \textbf{PCLAB} предназначена  для исследования  производительности
x86 совместимых ЭВМ  cIA32 архитектурой, работающих под управлением
операционной системы Windows (версий 95 и старше). Исследование организации ЭВМ
заключается в проведении ряда экспериментов, направленных на построение
зависимостей  времениобработки критических участков кода от изменяемых
параметров. Набор реализуемых программой экспериментов позволяет исследовать
особенности построения современных подсистем памяти ЭВМ и процессорных
устройств, выявить конструктивные параметры конкретных моделей ЭВМ.

Процесс сбора и анализа экспериментальных данных в \textbf{PCLAB} основан на
процедуре профилировки критического кода, т.е. в измерении времени его
обработки центральным процессорным устройством. При исследовании конвейерных
суперскалярных процессорных устройств, таких как 32-х разрядные процессоры
фирмы Intel или AMD, способных выполнять переупорядоченную обработку
последовательности команд программы, требуется использовать специальные
средства измерения временных интервалов и запрещения переупорядочивания
микрокоманд. Для измерения времени работы циклов в \textbf{PCLAB} используется
следующая методика:
\begin{itemize}
    \item длительность   обработки   участка   профилируемой   программы
          характеризуется изменением величины счетчика тактов процессора,
          произошедшим за время его работы; 
    \item для предотвращения влияния соседних участков кода на результаты
          измерений, передначалом замера и после его окончания необходимо
          выдать команду упорядоченноговыполнения CPUID, препятствующую
          переупорядочивание потока команд на конвейере процессора;
    \item замеры количества тактов процессора необходимо повторить несколько
          раз;
    \item взаимное влияние последовательных повторов экспериментального участка
          программыисключается благодаря очищению кэш-памяти и буферов
          процессора;
    \item часть граничных результатов отбрасывается (как наибольших, так и
          наименьших).
\end{itemize}
