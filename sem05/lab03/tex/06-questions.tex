\chapter{Контрольные вопросы}

В данном разделе представлены ответы на контрольные вопросы.

\noindent
\begin{itemize}[left=-0.5\parindent]
    \item \textbf{Назовите причины расслоения оперативной памяти}

        При расслоении банки памяти обычно упорядочиваются так, чтобы N
        последовательных адресов памяти $i$, $i+1$, $i+2$, ..., $i + N-1$
        приходились на N различных банков. Можно достичь в N раз большей
        скорости доступа к памяти в целом, чем у отдельного ее банка, если
        обеспечить при каждом доступе обращение к данным в каждом из банков.
        Имеются разные способы реализации таких расслоенных структур.
        Большинство из них напоминают конвейеры, обеспечивающие рассылку
        адресов в различные банки и мультиплексирующие поступающие из банков
        данные. Таким образом, степень или коэффициент расслоения определяют
        распределение адресов по банкам памяти. Такие системы оптимизируют
        обращения по последовательным адресам памяти, что является характерным
        при подкачке информации в кэш-память при чтении, а также при записи, в
        случае использования кэш-памятью механизмов обратного копирования.
        Однако, если требуется доступ к непоследовательно расположенным словам
        памяти, производительность расслоенной памяти может значительно
        снижаться.
        ~\\

    \item \textbf{Как в современных процессорах реализована аппаратная
                 предвыборка?}

        Аппаратная предвыборка происходит неявно, без участия человека или
        компилятора. Кэш-контроллер анализирует, по каким адресам и в каком
        порядке программа обращается к оперативной памяти, пытается
        предугадать, какие данные вскоре могут понадобиться программе, и
        осуществляет их автоматическую предвыборку в кэш-память.

        Если кэш-контроллер обнаруживает, что оперативная память
        последовательно опрашивается с некоторым фиксированным шагом, то он
        делает предположение, что в программе имеет место некоторая
        последовательная обработка массива, и начинает загружать следующие
        блоки данных заранее, еще до того, как к ним реально произойдет
        обращение.

        Если кэш-контроллер распознал в последовательности обращений обход
        массива ошибочно, то данные, которые он загрузит в кэш-память, могут
        не понадобиться. Более того, они могут «вытеснить» какие-либо полезные
        данные, находившиеся в кэш-памяти, и их придется загружать, когда
        программа к ним обратится снова. Тем не менее в большинстве случаев
        аппаратная предвыборка данных в кэш-память сказывается на скорости
        работы программ положительно.
        ~\\

    \item \textbf{Какая информация хранится в TLB?}

        TLB хранит последние переводы виртуальной памяти в физическую память и
        может называться кэшем преобразования адресов.
        ~\\

    \item \textbf{Какой тип ассоциативной памяти используется в кэш-памяти
                  второго уровня современных ЭВМ и почему?}

        Используется множественно-ассоциативный тип в кэш-памяти второго
        уровня в современных ЭВМ, так как она устраняет проблему
        многочисленного вытеснения информации, возникающую в памяти с прямым
        отображением, и проблему поиска нужной строки, возникающую в полностью
        ассоциативной памяти.
        ~\\

    \item \textbf{Приведите пример программной предвыборки}

        В некотором итерационном вычислительном процессе разумно делать
        предвыборку в кэш-память тех данных, которые понадобятся на следующих
        итерациях.
\end{itemize}
