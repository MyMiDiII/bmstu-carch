\section{Характеристики процессора}

На рисунке \ref{img:settings} представлены характеристики процессора компьютера в лабороторной аудитории, которые получены в программе \textbf{PCLAB}.

%\imgScale{0.3}{settings}{Характеристики процессора}
%
%\imgScale{0.45}{ident_proc}{Идентификация процессора}


\chapter{Эксперименты}

Для исследования производительности был проведен ряд экспериментов, которые представлены ниже.


\section{Исследования расслоения динамической памяти}

\textbf{Цель эксперимента} -- определение   способа   трансляции   физического   адреса,используемого при обращении к динамической памяти.

\subsection{Исходные данные}
\begin{enumerate}
	\item Размер линейки кэш-памяти верхнего уровня.
	\item Объем физическойпамяти.
\end{enumerate}

\subsection{Результаты эксперимента}
\begin{enumerate}
	\item Количество банков динамической памяти.
	\item Размер одной страницы динамической памяти.
	\item Количество страниц в динамической памяти.
\end{enumerate}


\subsection{Описание проблемы}
В связи с конструктивной неоднородностью   оперативной памяти,   обращение к последовательно расположенным данным   требует различного времени. В связи с этим, для создания эффективных программ необходимо учитывать расслоение памяти, характеризуемое способом трансляции физического адреса. 

\subsection{Суть эксперимента}  
Для определения способа трансляции физического адреса приф ормировании сигналов выборки банка, выборки строки и столбца запоминающего массива применяется   процедура   замера   времени   обращения   к   динамической   памяти   по последовательным   адресам   с   изменяющимся   шагом   чтения.   Для   сравнения   времен используется обращение к одинаковому количеству различных ячеек,  отстоящих друг от друга   на   определенный шаг. Результат эксперимента представляется зависимостью времени (или количества тактов процессора), потраченного на чтение ячеек, от шага чтения. Для проведения эксперимента необходимо задать изменяемые параметры.


\subsection{Проведение эксперимента}

Изменяемые параметры.

\begin{enumerate}
	\item Максимальное расстояние между читаемыми блоками = 7.
	\item Шаг увеличения расстояния между читаемыми 4-х байтовыми ячейками = 10.
	\item Количество страниц в динамической памяти = 4.
\end{enumerate}


%\imgScale{0.3}{exp1}{Результат выполнения эксперимента в PCLAB}


\clearpage


Также получим значения следующих величин:

\begin{equation}
	B = T_1 / M,
\end{equation}
где \textit{B} -- количество банков; $T_1$ -- минимальный шаг чтения динамической памяти, при котором происходит постоянное обращение к одному и тому же банку; \textit{M} -- объем данных, являющийся минимальной порцией обмена кэш-памяти верхнего уровня с оперативной памятью и соответствует размеру линейки кэш-памяти верхнего уровня


А также

\begin{equation}
	S = T_2 / B,
\end{equation}
где \textit{B} -- количество банков; $T_2$ -- соответствует расстоянию (в байтах) между началом двух последовательных страниц одного банка; \textit{S} -- количество секторов.


В результате эксперимента было получено, что $T_1 = 128$ и $T_2 = 4096$, тогда:

\begin{equation}
	B = T_1 / M = 128 / 64 = 2;
\end{equation}

\begin{equation}
	S = T_2 / B = 4096 / 2 = 2048.
\end{equation}


\subsection{Вывод}

Необходимо учитывать расслоение памяти при обработке данных, потому что оперативная память неоднородна, поэтому для обращение к последовательно расположенным данным может потребоваться разное количество времени.



\section{Сравнение эффективности ссылочных и векторных структур}

\textbf{Цель эксперимента} -- оценка   влияния   зависимости   команд   по   данным   на эффективность вычислений.

\subsection{Результаты эксперимента}
Отношение времени работы алгоритма, использующего зависимые данные, ко времени обработки аналогичного алгоритма обработки независимых данных.

\subsection{Описание проблемы}
Обработка зависимых данных происходит в тех случаях, когда результат работы одной команды используется в качестве адреса операнда другой. При программировании на языках высокого уровня такими операндами являются указатели, активно используемые при обработке ссылочных структур данных: списков, деревьев,графов. Обработка данных структур процессорами с длинными конвейерами команд приводит к заметному увеличению  времени работы алгоритмов: адрес загружаемого операнда   становится   известным   только   после   обработки   предыдущей   команды.   В противоположность этому, обработка векторных структур, таких как массивы, позволяет эффективно использовать аппаратные возможности ЭВМ.

\subsection{Суть эксперимента}  
Для сравнения эффективности векторных и списковых структур в эксперименте применяется профилировка кода двух алгоритмов поиска минимального значения. Первый алгоритм использует для хранения данных список, в то время как во втором применяется массив. Очевидно, что время работы алгоритма поиска минимального значения в списке зависит от его фрагментации, т.е. от среднего расстояния междуэлементами   списка. 


\subsection{Проведение эксперимента}

Изменяемые параметры.

\begin{enumerate}
	\item Количество элементов в списке = 4.
	\item Максимальная фрагментации списка = 256.
	\item Шаг увеличения фрагментации = 4.
\end{enumerate}

На графике, который представлен на рисунке \ref{img:exp2}:
\begin{itemize}
	\item \textbf{красная линия} -- количество тактов работы алгоритма с использованием списка;
	\item \textbf{зеленая линия} -- количество тактов работы алгоритма с использованием массива.
\end{itemize}

При этом на оси абсцисс указана фрагментация списка.

%\imgScale{0.3}{exp2}{Результат выполнения эксперимента в PCLAB}
%
%\imgScale{0.3}{exp2_time}{Результат сравнения времени работы}

\clearpage


Также на рисунке \ref{img:exp2_time} представлен результат сравнения, на котором видно, что список обрабатывается в 23,071307 раз дольше массива.



\subsection{Вывод}

Необходимо учитывать технологический фактор решаемой задачи, от которой и зависит то, какая структура данных подхдоит больше. В данном случае приорететным является использование массива. 




\section{Исследование эффективности программной предвыборки}

\textbf{Цель эксперимента} -- оценка   влияния   зависимости   команд   по   данным   на эффективность вычислений.

\subsection{Исходные данные}
\begin{enumerate}
	\item Степень ассоциативности.
	\item Размер TLB данных.
\end{enumerate}

\subsection{Результаты эксперимента}
Отношение времени последовательной обработки блока данных ко времени обработки блока с применением предвыборки; время и количествотактов первого обращения к странице  данных.


\subsection{Описание проблемы}
Обработка больших массивов информации сопряжена с открытием большого количества физических страниц памяти.  При первом обращении к странице памяти наблюдается   увеличенное время доступа к данным. Это связано с необходимостью преобразования логического адреса в физический адрес памяти, а также c открытием   страницы   динамической   памяти   и   сохранения   данных   в   кэш-памяти. Преобразование выполняется на основе информации о использованных ранее страницах, содержащейся в TLB буфере процессора. Первое обращение к странице при отсутствии информации   в  TLB  вызывает двойное обращение к оперативной памяти: сначала за информацией из таблицы страниц, а далее за востребованными данными.   Предвыборка заключается   в   заблаговременном   проведении   всех   указанных   действий   благодаря дополнительному запросу небольшого количества данных из оперативной памяти.

\subsection{Суть эксперимента}  
Эксперимент основан на замере времени двух вариантов подпрограмм последовательного чтения страниц оперативной памяти. В первом варианте выполняется последовательное чтение без дополнительной оптимизации, что приводит к дополнительным двойным обращениям. Во втором варианте перед циклом чтения страниц используется   дополнительный   цикл   предвыборки,   обеспечивающий   своевременную загрузку информации в TLB данных. 


\subsection{Проведение эксперимента}

Изменяемые параметры.

\begin{enumerate}
	\item Шаг увеличения расстояния между читаемыми данными = 256.
	\item Размер массива = 256.
\end{enumerate}

На графике, который представлен на рисунке \ref{img:exp3}:
\begin{itemize}
	\item \textbf{красная линия} -- количество тактов работы алгоритма без использования предвыборки;
	\item \textbf{зеленая линия} -- количество тактов работы алгоритма с использованием предвыборки.
\end{itemize}


%\imgScale{0.3}{exp3}{Результат выполнения эксперимента в PCLAB}
%
%\imgScale{0.3}{exp3_time}{Результат сравнения времени работы}

\clearpage

Также на рисунке \ref{img:exp3_time} представлен результат сравнения, на котором видно, что обработка без загрузки страниц в TLB в 1.5124754 раза дольше.



\subsection{Вывод}

При использовании заблаговременной загрузки страниц можно получить ускорение работы программы.




\section{Исследование способов эффективного  чтения оперативной памяти}

\textbf{Цель эксперимента} -- исследование возможности ускорения вычислений благодаряиспользованию структур данных, оптимизирующих механизм чтения оперативной памяти.

\subsection{Исходные данные}
\begin{enumerate}
	\item Адресное расстояние между банками памяти.
	\item Размер буфера чтения.
\end{enumerate}

\subsection{Результаты эксперимента}
Отношение   времени   обработки   блока   памяти неоптимизированной структуры ко времени обработки блока структуры, обеспечивающей эффективную загрузку и параллельную обработку данных


\subsection{Описание проблемы}
При   обработке информации, находящейся в нескольких страницах и банках оперативной памяти возникают задержки, связанные с необходимостью открытия и закрытия страниц DRAM памяти. При программировании на языках высокого уровня такая ситуация наблюдается при интенсивной обработке нескольких массивов данных   или   обработке   многомерных   массивов.   При   этом   процессоры,   в   которых реализованы   механизмы   аппаратной   предвыборки,   часто   не   могут   организовать эффективную загрузку данных. Кроме этого, объемы запрошенных данных оказываются заметно меньше размера пакета, передаваемого из оперативной памяти. Таким образом,эффективная обработка нескольких векторных структур данных без их дополнительной оптимизации не использует в должной степени возможности аппаратных ресурсов. Для создания структур данных, оптимизирующих их обработку   современными процессорами, требуется максимально исключить несвоевременную передачу данных, т.е.передавать в каждом пакете только востребованную для вычислений информацию. В результате   такой   оптимизации   снижается   количество   кэш-промахов,   сокращается количество открытий и закрытий страниц  DRAM-памяти, обеспечивается параллельная обработка данных и выполнение операций загрузки и выгрузки.

\subsection{Суть эксперимента}  
Для сравнения производительности алгоритмов, использующих оптимизированные и неоптимизированные структуры данных используется профилировка кода двух подпрограмм, каждая из которых должна выполнить обработку нескольких блоков оперативной памяти. В алгоритмах обрабатываются двойные слова данных (4 байта), что существенно меньше размера пакета (32 – 128 байт). Неоптимизированный вариант структуры данных представляет собой несколько массивов в оперативной памяти, в то время как оптимизированная структура состоит из чередующихся данных каждого массива. 


\subsection{Проведение эксперимента}

Изменяемые параметры.

\begin{enumerate}
	\item Размер массива = 2.
	\item Количество потоков данных = 64.
\end{enumerate}

На графике, который представлен на рисунке \ref{img:exp4}:
\begin{itemize}
	\item \textbf{красная линия} -- количество тактов работы алгоритма, который использует неоптимизированную структуру;
	\item \textbf{зеленая линия} -- количество тактов работы алгоритма с использованием оптимизированной структуры.
\end{itemize}


%\imgScale{0.3}{exp4}{Результат выполнения эксперимента в PCLAB}
%
%\imgScale{0.3}{exp4_time}{Результат сравнения времени работы}


\clearpage


Также на рисунке \ref{img:exp4_time} представлен результат сравнения, на котором видно, что неоптимизированная структура обрабатывается в 1.6762066 раза дольше.


\subsection{Вывод}

Чем лучше упорядочены данные, тем быстрее работает алгоритм.




\section{Исследование конфликтов в кэш-памяти}

\textbf{Цель эксперимента} -- исследование   влияния   конфликтов   кэш-памяти   наэффективность вычислений.

\subsection{Исходные данные}
\begin{enumerate}
	\item Размер банка кэш-памяти данных первого и второго уровня.
	\item Степень ассоциативности кэш-памяти первого и второго уровн.
	\item Размер линейки кэш-памяти первого и второго уровня.
\end{enumerate}

\subsection{Результаты эксперимента}
Отношение времени обработки массива с конфликтами вкэш-памяти ко времени обработки массива без конфликтов.


\subsection{Описание проблемы}
Наборно-ассоциативная кэш-память состоит из линеек данных, организованных в несколько независимых банков. Выбор банка для каждой порции кэшируемых данных выполняется по ассоциативному принципу, т.е. из условия улучшения представительности выборки, в то время как целевая линейка в каждом из банков жестко определяется по младшей части физического адреса. Совокупность таких линеек всех банков принято называть набором. Таким образом, попытка читать данные из оперативной памяти с шагом, кратным размеру банка, приводит к их помещению в один и тот же набор.Если же количество запросов превосходит степень ассоциативности кэш-памяти, т.е.количество банков или количество линеек в наборе, то наблюдается постоянное вытеснение данных из кэш-памяти, причем больший ее объем  остается незадействованным.

\subsection{Суть эксперимента}  
Для определения степени влияния конфликтов в кэш-памяти на эффективность вычислений используется профилировка двух процедур чтения и обработки данных. Первая процедура построена таким образом, что чтение данных выполняется с шагом, кратным размеру банка. Это порождает постоянные конфликты в кэш-памяти.Вторая процедура оптимизирует размещение данных в кэш с помощью задания смещения востребованных данных на некоторый шаг, достаточный для выбора другого набора. Этотшаг соответствует размеру линейки.  


\subsection{Проведение эксперимента}

Изменяемые параметры.

\begin{enumerate}
	\item Размер банка кэш-памяти = 256.
	\item Размер линейки кэш-памяти = 32.
	\item Количество читаемых линеек = 64.
\end{enumerate}

На графике, который представлен на рисунке \ref{img:exp5}:
\begin{itemize}
	\item \textbf{красная линия} -- количество тактов работы функции, которая читает данные с конфилктами в кэш-памяти;
	\item \textbf{зеленая линия} -- количество тактов работы функции, которая не вызывает конфликтов в кэш-памяти.
\end{itemize}

При этом на оси абсцисс отражено смещение читаемой ячейки от начала блока данных.


%\imgScale{0.2}{exp5}{Результат выполнения эксперимента в PCLAB}

\clearpage


В результате работы сравнения было определено, что с конфликтами данных производится в 6.2300479 раза дольше.



\subsection{Вывод}

Использование кэш-памяти позволяет значительно ускорить работу процессора.




\section{Сравнение алгоритмов сортировки}

\textbf{Цель эксперимента} -- сследование способов эффективного использования  памяти ивыявление наиболее эффективных алгоритмов сортировки, применимых в вычислительных системах.

\subsection{Исходные данные}
\begin{enumerate}
	\item Количество процессоров вычислительной системы.
	\item Размер пакета.
	\item Количество элементов в массиве.
	\item Разрядность элементов массива.
\end{enumerate}

\subsection{Результаты эксперимента}
Отношение времени сортировки массива алгоритмом QuickSort ко времени сортировки алгоритмом Radix-Counting Sort  и ко времени сортировки Radix-Counting Sort, оптимизированной под 8-процессорную вычислительную систему.


\subsection{Описание проблемы}
Существует несколько десятков алгоритмов сортировки. Их можно классифицировать по таким критериям, как: назначение (внутренняя и внешняя сортировки), вычислительная сложность (алгоритмы с вычислительными сложностями O(n2), O(n*log(n)), O(n), O(n/log(n))), емкостная сложность (алгоритмы, требующие и нетребующие   дополнительного   массива),   возможность   распараллеливания   (нераспараллеливаемые, ограниченно распараллеливаемые, полностью распараллеливаемые), принцип   определения   порядка   (алгоритмы,   использующие   парные   сравнения   и   неиспользующие парные сравнения).

\subsection{Суть эксперимента}  
Эксперимент основан на замере времени трех вариантов алгоритмов сортировки (Quick Sort, Radix-Counting Sort, Оптимизированный Radix-CountingSort) .


\subsection{Проведение эксперимента}

Изменяемые параметры.

\begin{enumerate}
	\item Количество 64-х разрядных элементов массивов = 2.
	\item Шаг увеличения размера массива = 128.
\end{enumerate}

На графике, который представлен на рисунке \ref{img:exp6}:
\begin{itemize}
	\item \textbf{фиолетовая линия} -- количество тактов работы алгоритма QuickSort;
	\item \textbf{красная линия} -- количество тактов работы алгоритма неоптимизированного алгоритма Radix-Counting;
	\item \textbf{зеленая линия} -- количество тактов работы оптимизированного под 8-процессорную вычислительную систему алгоритма Radix-Counting.
\end{itemize}

При этом на оси абсцисс отражено количество 64-разрядных элементов сортируемых массивов, а на оси ординат -- количество тактов.


%\imgScale{0.2}{exp6}{Результат выполнения эксперимента в PCLAB}
%
%\imgScale{0.6}{exp6_time}{Результат сравнения времени работы}

Также на рисунке \ref{img:exp6_time} представлен результат сравнения, на котором видно, что QuickSort работал в 1.95434 раза дольше, чем Radix-Counting Sort, и в 2.353913 раза дольше, чем Radix-Counting Sort, оптимизированного под 8-процессорную ЭВМ.


\clearpage



\subsection{Вывод}

Radix-Counting Sort является более быстрой сортировкой, чем QuickSort, при этом даже Radix-Counting Sort можно улучшить, оптимизировав его под 8-процессорную ЭВМ.
