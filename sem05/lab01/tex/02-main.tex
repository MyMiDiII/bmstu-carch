\section*{Схема разрабатываемой СНК}

В данной лабораторной работе реализовывается система на кристалле состоящая из:
\begin{itemize}[left=\parindent]
    \item микропроцессорного ядра Nios II/e;
    \item внутренней оперативной памяти СНК;
    \item системной шины Avalon;
    \item блока синхронизации и сброса;
    \item блока идентификации версии проекта;
    \item контроллера UART (интерфейс RS232).
\end{itemize}

Функциональная схема разрабатываемой системы на кристалле представлена на
рисунке 1.

\img{70mm}{img01.jpg}{Функциональная схема разрабатываемой системы на
                      кристалле}

\img{60mm}{img02.png}{Готовый модуль в системе проектирования систем на
                      кристалле}

\img{19mm}{img03.png}{Таблица распределения адресов модулей в системе на
                      кристалле}

\newpage

\begin{lstinputlisting}[
	caption={Код эхо-программы приема-передачи по интерфейсу RS232},
	label={lst:echo},
	linerange={1-16}
]{../data/listings/hello.c}
\end{lstinputlisting}

\img{35mm}{img04.jpg}{Результаты тестирования эхо-программы}

\newpage

\begin{lstinputlisting}[
	caption={Код программы, передающей по UART значение SystemID},
	label={lst:group},
	linerange={1-23}
]{../data/listings/group.c}
\end{lstinputlisting}

