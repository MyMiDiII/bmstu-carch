\section*{Схема разрабатываемой СНК}

В данной лабораторной работе реализовывается система на кристалле состоящая из:
\begin{itemize}[left=\parindent]
    \item микропроцессорного ядра Nios II/e;
    \item внутренней оперативной памяти СНК;
    \item системной шины Avalon;
    \item блока синхронизации и сброса;
    \item блока идентификации версии проекта;
    \item контроллера UART (интерфейс RS232).
\end{itemize}

Функциональная схема разрабатываемой системы на кристалле представлена на
рисунке 1.

\img{70mm}{img01.jpg}{Функциональная схема разрабатываемой системы на
                      кристалле}

\newpage
\section*{Проектирование CНК в Quartus II}

Модуль системы на кристалле в Quartus II создается в полном соответствии с выше
приведенной схемой: добавляются необходимые компоненты системы и строятся связи
между ними через шину Avalon, --- в чем можно убедиться, внимательно сравнив
схему, представленную на рисунке 1, и готовый модуль в системе проектирования
(рисунок 2).


\img{60mm}{img02.png}{Готовый модуль в системе проектирования систем на
                      кристалле}

Для правильно работы системы необходимо корректное распределение адресного
пространства между устройствами, так как неверное обращение к памяти может
привести к непредсказуемому поведению системы или даже полной
неработоспособности ПЛИС. В Quartus II данное распределение происходит
автоматически, и, как видно из рисунка 3, каждому подключенному компоненту
выделяется свое адресное пространство, при этом данные и код каждого устройства
имеют одинаковое адресное пространство, что соответсвует одному из принципов
архитектуры фон Неймана.

\img{19mm}{img03.png}{Таблица распределения адресов модулей в системе на
                      кристалле}

\newpage
\section*{Программное обеспечение}

После проектирования и сборки аппаратного обеспечения можно переходит к
написанию программного кода. В данной лабораторной работе реализуются
простейшие функциональности. Одной из них является отклик СНК на нажатие
клавиши на клавиатуре. Код такой эхо-программы представлен на листинге 1.
Результаты тестирования приведены на рисунке 4.

\begin{lstinputlisting}[
	caption={Код эхо-программы приема-передачи по интерфейсу RS232},
	label={lst:echo},
	linerange={1-16}
]{../data/listings/hello.c}
\end{lstinputlisting}

\img{35mm}{img04.jpg}{Результаты тестирования эхо-программы}

В виде более сложность функциональности нам предлагалось реализовать передачу
по UART значения SystemID в виде четырех байт символов в формате ASCII.

Для проверки работоспособности программы в поле SystemID (рисунок 5) было
записано значение в шестнадцаричной системе счисления, соответсвующее
десятичному значению, полученному конкатенацией номера группы и варианта
($5302_{10} = 14B6_{16}$).

\img{15mm}{img05.png}{Запись необходимого значения в SystemID}

\newpage
Реализованная нами программа представлена на листинге 2.

\begin{lstinputlisting}[
	caption={Код программы, передающей по UART значение SystemID},
	label={lst:group},
	linerange={1-23}
]{../data/listings/group.c}
\end{lstinputlisting}

