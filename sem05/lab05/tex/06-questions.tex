\chapter{Контрольные вопросы}

В данном разделе представлены ответы на контрольные вопросы.

\noindent
\begin{itemize}[left=-0.5\parindent]
    \item \textbf{Назовите преимущества и недостатки аппаратных ускорителей на
          ПЛИС по сравнению с CPU и графическими ускорителями}

        Преимущества:
        \begin{itemize}
            \item создание специализированной вычислительной структуры для
                  реализации желаемой функциональности;
            \item низкая стоимость в сравнению с обычными аппаратными
                  ускорителями;
            \item большая частота эмуляции;
            \item компактность.
        \end{itemize}

        Недостатки:
        \begin{itemize}
            \item необходимость перекомпиляции проекта и переконфигурации ПЛИС
                  при любом исправлении содержимого проекта;
            \item наличие специализированного программного обеспечения для
                  разделения модели микросхемы на части для загрузки в
                  отдельные ПЛИС.
        \end{itemize}
        ~\\

    \item \textbf{Назовите основные способы оптимизации циклических конструкций
          ЯВУ, реализуемых в виде аппаратных ускорителей}
        \begin{itemize}[leftmargin=\parindent]
            \item конвейерная обработка;
            \item разворачивание циклов;
            \item потоковая обарботка.
        \end{itemize}
        ~\\

    \item \textbf{Назовите этапы работы программной части ускорителя в хост
          системе}
        \begin{enumerate}[leftmargin=\parindent]
            \item[Этап 1.] Инициализация среды OpenCL.
            \item[Этап 2.] Создание трех буферов, необходимых для
            обмена данными с ядром: два буфера для передачи исходных данных и
            один для вывода результата.
            \item[Этап 3.] Запуск задачи на исполнение.
            \item[Этап 4.] Чтение выходного буфера, содержащего результаты
            работы ядра, после завершения работы всех команд.
        \end{enumerate}
        ~\\

    \item \textbf{В чем заключается процесс отладки для вариантов сборки
          Emulation-SW, Emulation-HW и Hardware?}
        \begin{itemize}[leftmargin=\parindent]
            \item \textit{Программная эмуляция} (Emulation-SW) --- код ядра
                  компилируется для работы на ЦПУ хост-системы. Этот вариант
                  сборки служит для верификации совместного исполнения кода
                  хост-системы и кода ядра, для выявления синтаксических
                  ошибок, выполнения отладки на уровне исходного кода ядра,
                  проверки поведения системы.

            \item \textit{Аппаратная эмуляция} (Emulation-HW) --- код ядра
                  компилируется в аппаратную модель (RTL), которая запускается
                  в специальном симуляторе на ЦПУ. Этот вариант сборки и
                  запуска занимает больше времени, но обеспечивает подробное и
                  точное представление активности ядра. Данный вариант сборки
                  полезен для тестирования функциональности ускорителя и
                  получения начальных оценок производительности.

            \item \textit{Аппаратное обеспечение} (Hardware) - код ядра
                  компилируется в аппаратную модель (RTL), а затем реализуется
                  на FPGA. В результате формируется двоичный файл xclbin,
                  который будет работать на реальной FPGA.
            \end{itemize}
            ~\\

    \item \textbf{Какие инструменты и средства анализа результатов синтеза
          возможно использовать в Vitis HLS для оптимизации ускорителей?}
        \begin{itemize}[leftmargin=\parindent]
            \item компилятор Xilinx Vitis v++;
            \item Assistant View (получение отчетов о сборке аппартных ядер);
            \item внутрисхемный отладчик Vivado (отслеживание любых сигналов
                  ускорителя для анализа событий);
            \item сводный отчет Link Summary (Vitis Analyzer, получение
                  диаграмм системы и платформы и др.)
        \end{itemize}
\end{itemize}
