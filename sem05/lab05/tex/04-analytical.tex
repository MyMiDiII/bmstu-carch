\chapter{Основные теоретические сведения}

В данном разделе описаны методология ускорения вычислений на основе ПЛИС и способы
оптимизации времени обработки и пропускной способности на ускорителе
вычислений \textbf{Xilinx Alveo}.

\section{Методология ускорения вычислений на основе ПЛИС}

Ускорение вычислительных алгоритмов с использованием программируемых логических
интегральных схем (ПЛИС) имеет ряд преимуществ по сравнению с их реализацией на
универсальных микропроцессорах, или графических процессорах. В то время, как
традиционная разработка программного обеспечения связана с программированием на
заранее определенном наборе машинных команд, разработка программируемых
устройств --- это создание специализированной вычислительной структуры для
реализации желаемой функциональности.

Микропроцессоры и графические процессоры имеют предопределенную архитектуру с
фиксированным количеством ядер, набором инструкций, и жесткой архитектурой
памяти, и обладают высокими тактовыми частотами и хорошо сбалансированной
конвейерной структурой. Графические процессоры масштабируют производительность
за счет большого количества ядер и использования параллелизма SIMD/SIMT
(рисунок \ref{img:teor}). В отличие от них, программируемые устройства
представляют собой полностью настраиваемую архитектуру, которую разработчик
может использовать для размещения вычислительных блоков с требуемой
функциональностью. В таком случает, высокий уровень производительности
достигается за счет создания длинных конвейеров обработки данных, а не за счет
увеличения количества вычислительных единиц. Понимание этих преимуществ
является необходимым условием для разработки вычислительных устройств и
достижения наилучшего уровня ускорения.

\img{70mm}{teor}{Принципы организации вычислений на различных платформах}{teor}

Методологию создания ускорителей на ПЛИС с применением средств синтеза высокого
уровня (High Level Synthesis, HLS) можно представить в виде трех этапов:

\begin{itemize}
	\item создание архитектуры приложения;
	\item разработка ядра аппаратного ускорителя на языках C/C++;
	\item анализ производительности и выявление способов ее повышения.
\end{itemize}


\section{Оптимизация времени обработки и пропускной способности}

Существует несколько подходов к оптимизации.

\subsection{Конвейерная обработка циклов}

Полагая, что одна итерация цикла занимает более одного такта (фаза выборки
данных, фаза вычисления, фаза записи результата) в таком варианте может быть
организован конвейер выполнения. Для этого необходимо поместить в тело цикла
прагму \texttt{<HLS PIPELINE>}.

Без конвейерной обработки каждая последующая итерация цикла начинается в каждом
третьем такте. При конвейерной обработке цикл может начинать последующие
итерации цикла менее чем за три такта, например, в каждом втором такте или в
каждом такте. Для указания конкретного значения интервала запуска заданий
используется свойство Initiation inteval (II). Например: \texttt{\#pragma HLS
PIPELINE II = 1}

Стоит отметить, что конвейерная обработка цикла приводит к разворачиванию любых
циклов, вложенных внутрь конвейерного цикла. Если внутри цикла существуют
зависимости по данным, может оказаться невозможным достичь запуска новой
итерации в каждом такте, и результатом может быть больший интервал инициации.


\subsection{Разворачивание циклов}

Разворачивание циклов является общепризнанным механизмом снижения времени
выполнения циклов. Этот механизм может быть описан самим разработчиком вручную,
если он просто повторит вычисления и сократит количество итераций цикла. С
помощью прагмы \texttt{<HLS UNROLL>} компилятор \texttt{v++} позволяет
запустить механизм автоматического разворачивания цикла, частично или
полностью.


\subsection{Потоковая обработка}

Реализация механизма потоковой обработки (\texttt{\#pragma HLS DATAFLOW}) также
опирается на представление вычислительных действий в виде многостадийного
конвейера. Однако, в то время как директива конвейеризации (\texttt{\#pragma
HLS PIPELINE}) используется для реализации конвейера выполнения операций внутри
функций или циклов, потоковая обработка данных позволяет сформировать конвейер
из более крупных вычислительных блоков: нескольких функций или нескольких
последовательных циклических конструкций. Таким образом, директива \texttt{HLS
DATAFLOW} позволяет сформировать вычислительный конвейер на уровне задач. Для
этих целей компилятор HLS Vitis выполнит анализ зависимостей по данным между
задачами и реализует структуру обрабатывающих блоков и FIFO очередей между
ними.
