В основе построения современных ЭВМ лежат принципы Фон-Неймана:

\textbf{Двоичное кодирование информации}

Этот принцип заключается в том, что данные и команды кодируются двоичными
цифрами 0 и 1.  Информация представляется в двоичном виде и имеет свой формат.
Последовательность битов в формате, имеющая определенный смысл называют полем.
Так, в формате числа обычно выделяют поле знака и поле значащих разрядов, в
формате команды -два поля: поле кода операции (какая операция должна быть
выполнена) и поле адресов (в зависимости от типа команды).

\textbf{Программное управление}

Все выполняемые действия должны быть представлены в виде программы, состоящей
из команд - последовательности управляющих слов. Команда представляет собой
операцию из набора операций, реализуемых вычислительной машиной. Команды
программы хранятся в последовательных ячейках памяти и выполняются в порядке их
расположения в программе. 

\textbf{Адресность памяти}

Структура основной памяти состоит из пронумерованных ячеек, причем в любой
момент процессор имеет доступ к любой ячейке. Двоичные коды команд и данных
разделяются на единицы информации, которые называют словами. Они хранятся в
ячейках памяти. Доступ к командам и данным осуществляется при помощи номеров
соответствующих ячеек — адресов.

\textbf{Однородность памяти}

Суть данного принципа заключается в том, что команды и данные хранятся в одной
и той же памяти и внешне в памяти неразличимы. Распознать их можно только по
способу использования. Это позволяет производить над командами операции,
которые производятся над числами. Концепция единой памяти для хранения команд и
данных принята для вычислительных машин в Принстонском университете и названа
принстонской архитектурой, в то время как в Гарвардском университете
реализовывалась идея отдельной памяти команд и отдельной памяти данных, такой
вид архитектуры назвали гарвардской.

\textbf{Базовая схема работы вычислительной машины (ВМ)}: информация поступает из
подсоединенных к ВМ устройств ввода, результаты вычислений выводятся на
устройства вывода. Чтобы программа могла выполняться, команды и данные должны
располагаться в основной памяти. Устройство управления отвечает за извлечение и
исполнение команд  и координацию устройств ВМ. Обрабатывающее устройство
обеспечивает арифметическую и логическую обработку двух входных переменных
(операндов), в итоге которой формируется выходная переменная (результат).

\img{50mm}{img01}{}{img01}

\textbf{Принципы микропрограммного управления.}

\img{50mm}{img02}{}{img02}

Принцип заключается в разделении устройства на операционные устройства и
управляющие устройства. 

Управляющее устройство (УУ) управляет всем процессом обработки. На него
поступают операнды, УУ анализирует состояние операционного устройства. По
состоянию линий управления в каждом такте УУ выдает микрокоманды. Их
совокупность называется микропрограммой.

\textbf{Принцип конвейерной обработки}

Конвейерная обработка представляет собой процесс, при котором сложные действия
разделяются на более короткие стадии. 

Характеристики конвейера:
\begin{itemize}[left=\parindent]
\item Пропускная способность - количество команд в единицу времени, которое он
может выполнить;
\item Латентность обработки - время, которое каждая команда тратит на
выполнение.
\end{itemize}

Преимущества:
\begin{itemize}[left=\parindent]
\item Увеличение пропускной способности;
\item Можно запустить большее количество команд на исполнение;
\item Более полное использование аппаратных ресурсов.
\end{itemize}

Использование: в памяти - 3 стадии конвейера, в процессоре - от 3 до 20.
