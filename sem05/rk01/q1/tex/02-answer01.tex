Идея наборно-ассоциативного способа отображения заключается в том, чтобы
объединить несколько кэшей с прямым размещением.  Данный метод имеет
отличительную особенность от других методов отображения ОП в кэш ---
использование нескольких банков, но строка в этих банках четко определяется
адресом. На схеме (рисунок \ref{img:img01}) банк --- кэш с прямым размещением.
Эти банки задаются параллельно, обычно, по 4 или 8 штук. При этом данные
размещаются в одном банке, если есть конфликтующие данные, они размещаются в
следующем банке (можно также расположить данные и по 4 банкам, тогда количество
конфликтов будет сокращено в 4 раза, 8 банков --- в 8 раз).  Также выделяется
набор для любой ячейки памяти --- в нем можно выбрать одну из имеющихся
кэш --- линеек. В двух ассоциативной памяти --- 2 линейки, в четырех
ассоциативной --- 4 линейки.

\img{90mm}{img01}{Наборно-ассоциативная кэш-память}{img01}

Достоинства данного способа:
\begin{itemize}[left=\parindent]
    \item аппаратная простота --- один компаратор на банк; банки объединяются в
        набор, при этом получается один дешифратор на все банки;
    \item ассоциативность обеспечивается за счет возможности выбора любого
        набора из них для хранения данных. 
\end{itemize}

Чтобы отследить, какая ячейка должна быть вытеснена при желании добавить что-то
в уже заполненный набор, нужно добавить дополнительную информацию к набору.
Иными словами, нужно хранить служебную информацию о состоянии каждой линейки
--- информацию о ее востребованности (нужно для построения алгоритма замещения)
или информацию о модификации данных (поскольку кэш память может по-разному
вести себя с данным, которые приходят от процессора, приходится по-разному ее
строить).  Наборно-ассоциативный способ отображения --- самый часто
используемый способ. Его преимущество заключается в возможности гибкого
изменения --- создания большего количества наборов или банков (по вертикали или
горизонтали).
