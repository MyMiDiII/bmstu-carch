\textbf{FPM DRAM} (Fast Page Mode Dynamic Random Access
Memory) --- динамическая память с быстрым постраничным
доступом. Принцип её работы основан на предположении о том, что
при обращении к какой-то ячейке, следующее обращении будет, скорее всего,
происходить к соседней ячейке, расположенной в пределах той же строки. Опишем
диаграмму работы FPM DRAM памяти.

Рассмотрим диаграмму работы FPM DRAM памяти, представленную на рисунке
\ref{img:img01}, в сравнении с DRAM памятью, представленную на рисунке
\ref{img:img02}. Так мы сможем в большей мере оценить достоинства и недостатки
FPM DRAM памяти.

\img{60mm}{img01}{Диаграмма работы FPM DRAM памяти}{img01}

\img{60mm}{img02}{Диаграмма работы DRAM памяти}{img02}

На рисунках \ref{img:img01}-\ref{img:img02} представлены верменные диаграммы
следующих сигналов: \textit{RAS} -- синхронизация обработки адреса строки;
\textit{CAS} -- синхронизация обработки адреса столбца; \textit{WE} -- сигнал
разрешения записи, --- и следующих линий: \textit{A} -- линия адреса;
\textit{D} -- линия данных.

И так, в начале работы по сбросу сигнала $\overline{RAS}$ происходит открытие и
усиленная регенерация строки, на это требуется время $t_{RCD}$, после завершения
этого процесса проиходит сбос сигнала $\overline{CAS}$ и чтения адреса столбца, и
далее данных.

Далее наблюдаются различия в диаграммах. FPM DRAM память поддерживает
сокращенные адреса, то есть, если запрашиваемая ячейка памяти находится в той
же самой строке, что и предыдущая повторная передача адреса строки уже не
требуется. Таким образом, если в DRAM памяти происходит восстановление
$\overline{RAS}$ и последующее повторение операции открытия и усиленной
регенерации строки с затратой на это дополнительного времени $t_{RAS}$, то в
FPM DRAM памяти этот сигнал остается быть равным 0, сообщая системе о том, что
следующая ячейка находится в той же строке и операция открытия и усиленной
регенерации не требуется, таким образом, на диаграмме FMP DRAM памяти
происходит только переадча адреса столбца, восстановлением и повторным сбросом
сигнала $CAS$.

Таким образом, получается, что на последовательную выборку данных из FPM DRAM
памяти затрачивается меньшее, чем $t_{RCD}+r_{CAC}$ время, которое затрачивалось в
DRAM памяти. Это дает выигрыш по скорости примерно в 2 раза. При этом при
обращении к разным строкам, FPM DRAM память работает полностью аналагично DRAM
памяти и не дает никакого выигрыша. Однако, так как по экспериментальные данные
доказывают, что наиболее часто происходит обращение к последовательным адресам,
FPM DRAM память работает эффективнее.
