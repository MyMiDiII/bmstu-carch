\chapter{Основные теоретические сведения}

В данном разделе будут описаны технология разработки ускорителей вычислений на
модулях Xilinx Alveo, а также архитектура разрабатываемого усорителя.

\section{Технология разработки ускорителей вычислений на модулях Xilinx Alveo}

Ускорителями вычислений принято называть специальные аппаратные устройства,
способные выполнять ограниченный ряд задач с большей параллельностью и за
меньшее время в сравнении с универсальными микропроцессорными ЭВМ. Как
правило, ускоритель представляет собой структуру, включающую большое количество
примитивных микропроцессорных устройств, объединенных шинами связей.

Создание ускорителей вычислений является трудоемким процессом, так как
охватывает не только аппаратную разработку самого устройства, но и предполагает
оптимизацию архитектуры ЭВМ для обеспечения наибольшей пропускной способности
каналов передачи операндов и результатов, а также минимизации задержек и
вычислительных затрат при ожидании работы ускорителей. Можно условно разделить
ускорители на два класса: ускорители на основе СБИС и на основе ПЛИС.

В данной лабораторной работе мы изучим технологию создания ускорителей
вычислений на основе ПЛИС (ускоритель \textbf{Xilinx Alveo U200} на основе ПЛИС
xcu200-fsgd2104-2-e архитектуры Xilinx UltraScale).

Для работы с ускорительной платой разработано специальное окружение
\textbf{XRT} (Xilinx Runtime), включающее компоненты пользовательского
пространства и драйвера ядра.

В оборудовании, используемом для проведения лабораторной работы, использована
так называемая XDMA сборка XRT, которая предполагает следующий сценарий
взаимодействия ускорителя и пользовательского ПО:

\begin{enumerate}
    \item Пользовательское ПО сканирует и инициализирует доступные
ускорительные платы, совместимые с XRT, определяет доступные ресурсы, создает
программное окружение пользовательского аппаратного ядра ускорителя (далее
используется термин kernel).
    \item Ресурсы локальной памяти ускорительной платы отображаются в
пространство памяти хост системы.
    \item Инициализируются каналы DMA для прямого доступа к памяти ускорителя.
    \item Данные, подлежащие обработке, копируются из ОЗУ в локальную память
ускорителя посредством DMA.
    \item Ядру ускорителя (или нескольким ядрам) посредством записи управляющих
регистров, передаются параметры вычислений. Пользователь может увеличивать
количество параметров по своему усмотрения. Типичным случаем является передача
указателей на начало буферов исходных операндов и буфера результата, а также
количество обрабатываемых значений.
    \item Хост-система выдает сигнал Start ядрам ускорителей, после чего
начинается обработка внутри платы Xilinx Alveo.
    \item По завершении обработки kernel устанавливает флаг DONE, что вызывает
прерывание по шине PCIe.
    \item Драйвер обрабатывает прерывание и сообщает пользовательскому ПО о
завершении обработки.
    \item Пользовательское ПО инициализирует DMA передачу результатов из
локальной памяти ускорителя в ОЗУ хост-системы.
\end{enumerate}

\section{Описание архитектуры разрабатываемого ускорителя}

В ходе лабораторной работы будет использован базовый шаблон так называемого RTL
проекта VINC, который может быть создан в IDE Xilinx Vitis и САПР Xilinx
Vivado. Шаблон VINC выполняет попарное сложение чисел исходного массива и
сохраняет результаты во втором массиве. Проект VINC включает:

\begin{itemize}
    \item Проект ПО хоста, выполняющий инициализацию аппаратного ядра и его
тестирование через OpenCL вызовы.
    \item Синтезируемый~~RTL~~проект~~ядра~~ускорителя~~на~~языках~~Verilog~~и
SystemVerilog.
    \item Функциональный тест ускорителя VINC на языке SystemVerilog.
\end{itemize}

Проект VINC представляет собой аппаратное устройство, связанное шиной AXI4 MM
(Memory mapped) с DDR[i] памятью, и получающее настроечные параметры по
интерфейсу AXI4 Lite от программного обеспечения хоста (рисунок
\ref{img:funcScheme}). В рамках всей системы используется единое 64-х разрядное
адресное пространство, в котором формируются адреса на всех AXI4 шинах.

\img{70mm}{funcScheme}{Функциональная схема разрабатываемой аппаратной
системы}{funcScheme}

В каждой карте U200 имеется возможность подключить ускоритель к любому DDR[i]
контроллеру в том регионе, где будет размещен проект. Всего для пользователя
доступны 3 динамических региона: SLR0,1,2, для которых выделены каналы
локальной памяти DDR[0], DDR[2], DDR[3] соответственно. Вся подключенная память
DDR[0..3] доступна со стороны статического региона, в котором размещена
аппаратная часть XRT.

Выбор одного из регионов для размещения проектов осуществляется на этапе так
называемой линковки конфигурационного файла при помощи компилятора
v++(фактически: компоновки, размещение и трассировки нескольких проектов в
единый конфигурационный файл).
