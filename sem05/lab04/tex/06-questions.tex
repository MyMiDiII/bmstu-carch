\chapter{Контрольные вопросы}

В данном разделе представлены ответы на контрольные вопросы.

\noindent
\begin{itemize}[left=-0.5\parindent]
    \item \textbf{Преимущества и недостатки XDMA и QDMA платформ}

        Сборка QDMA, доступная на картах ускорителей Alveo, предоставляет
        разработчикам прямое потоковое соединение с низкой задержкой между
        хостом и ядрами. Оболочка QDMA включает высокопроизводительный DMA,
        который использует несколько очередей, оптимизированных как для
        передачи данных с высокой пропускной способностью, так и для передачи
        данных с большим количеством пакетов.  Только QDMA позволяет передавать
        поток данных непосредственно в логику FPGA параллельно с их обработкой.

        Оболочка XDMA требует, чтобы данные сначала были полностью перемещены
        из памяти хоста в память FPGA, прежде чем логика FPGA сможет начать
        обработку данных, что влияет на задержку на запуска задачи.

        Потоковая передача напрямую в работающие ускорительные ядра позволяет
        быстро и без излишней буферизации передавать операнды и результаты
        вычислений на хост по потоковому интерфейсу AXI4 Stream. Решение QDMA
        подходит для приложений, в которых вычисления строятся на передачи
        сравнительно небольших пакетов, но при этом требуется высокая
        производительность и минимальная задержка отклика.
        ~\\

    \item \textbf{Последовательность действий, необходимых для инициализации
        ускорителя со стороны хост-системы}
        \begin{enumerate}[leftmargin=\parindent]
            \item Сканирование и инициализация доступных ускорительных план,
                  совместымых с XRT.
            \item Определение доступных ресурсов.
            \item Создание программного окружения пользовательского аппаратного
                  ядра ускорителя.
            \item Инициализация локальной памяти ускорителя посредством DMA.
            \item Передача параметров вычислений ядру ускорителя.
            \item Подача сигнала Start для начала обработку внутри платы.
            \item Инициализация DMA передачи результатов из локальной памяти
                  ускорителя в ОЗУ хост-системы по окончанию обработки.
        \end{enumerate}
        ~\\

    \item \textbf{Какова процедура запуска задания на исполнения в
        ускорительном ядре VINC}
        \begin{enumerate}[leftmargin=\parindent]
            \item Копирование данных из .xclbin и данных, подлежащих обработке,
                  в локальную память ускорителя посредством DMA.
            \item Создание исполняемого файла в памяти ускорителя.
            \item Получение параметров вычислений.
            \item Начало обработки по сигналу Start от хост-системы.
            \item Обработка.
            \item Установка флага DONE для сообщения хочт-системе о завершении
                  обработки.
        \end{enumerate}
        ~\\

    \item \textbf{Процесс линковки на основании содержимого файла v++\_*.log}
        \begin{enumerate}[leftmargin=\parindent]
            \item Анализ профиля устройства.
            \item Анализ конфигурационного файла.
            \item Поиск необходимых интерфейсов.
            \item Создание графа связности системы.
            \item Связывание~~синтезированных~~ядер~~с~~платформой~~(FPGA linking
                  synthesized kernels to platform).
            \item Оптимизация логики ПЛИС (FPGA logic optimization).
            \item Размещение логического блока в динамическом регионе (FPGA
                  logic placement).
            \item Маршрутизация ПЛИС (FPGA routing).
            \item Генерация файла *.xclbin.
        \end{enumerate}
\end{itemize}
