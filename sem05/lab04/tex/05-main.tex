\chapter{Практическая часть}

В данном разделе представлен ход выполнения лабораторной работы.

\section{Исходный проект VINC}

В данном подразделе описывается работа с исходным проектом VINC, код инкремента
данных в котором представлен на рисунке \ref{img:inc01}.

\img{15mm}{inc01}{Код инкремента данных}{inc01}

\subsection{Моделирование}

На рисунке \ref{img:diag01} рпредставлена транзация чтения данных вектора на
шине AXI4 MM из DDR памяти, на рисунке \ref{img:diag02} -- транзация записи
данных, на рисунке \ref{img:incDiag} -- инкремент данных в модуле.

\img{70mm}{diag01}{Транзакция чтения данных вектора на шине AXI4 MM из DDR
памяти}{diag01}

\img{60mm}{diag02}{Транзакция записи данных на шине AXI4 MM}{diag02}

\img{70mm}{incDiag}{Инкремент данных в модуле}{incDiag}
~\\
~\\
~\\

\section{Проект VINC по варианту}

В данном подразделе описывается работа с проектом VINC по индивидуальному
варианту, код инкремента данных в котором представлен на рисунке
\ref{img:vInc01}.

\img{15mm}{vInc01}{Код инкремента данных}{vInc01}

\subsection{Моделирование}

На рисунке \ref{img:vDiag01} рпредставлена транзация чтения данных вектора на
шине AXI4 MM из DDR памяти, на рисунке \ref{img:vDiag02} -- транзация записи
данных, на рисунке \ref{img:vIncDiag} -- инкремент данных в модуле.

\img{60mm}{vDiag01}{Транзакция чтения данных вектора на шине AXI4 MM из DDR
памяти}{vDiag01}

\img{60mm}{vDiag02}{Транзакция запписи данных на шине AXI4 MM}{vDiag02}

\img{60mm}{vIncDiag}{Инкремент данных в модуле}{vIncDiag}

\subsection{Сборка проекта}

Для сборки проекта компилятором \texttt{v++} используется конфигурационный файл
\textit{*.cfg}, который содержит основную информацию для работы компилятора:

\begin{itemize}
    \item количество и условные имена экземпляров ядер;
    \item тактовая частота работы ядра;
    \item для каждого ядра: выбор региона SLR, памяти DDR,
          высокопроизводительной памяти PLRAM;
    \item параметры синтеза и оптимизации проекта.
\end{itemize}

На листинге \ref{lst:conf} представлен конфигурационный файл использующий в
данной работе в соответствии с индивидуальным вариантом.

\mylisting{Конфигурационный файл}{conf}{1-11}{rtl_kernel_wizard_0_ex.cfg}

В результате компиляции генерируется файл \texttt{*.xclbin}, который может быть
передан в ускорительную карту, также генерируется лог файл \texttt{v++*.log} и
файл описания ресурсов \texttt{*.xclbin.info}, содержимое которых приведено в
приложении \hyperlink{apA}{А}.

\subsection{Тестирование}

Для тестирования используется программа, исходный код которой представлен в
файле \texttt{host\_example.cpp}. Цикл проверки результатов работы ускорителя по
индивидуальному варианту представлен на листинге \ref{lst:host}.

\mylisting{Модифицированный модуль
\texttt{host\_example.cpp}}{host}{313-318}{host_example.cpp}

Результаты тестирования представлены на рисунке \ref{img:test}.

\img{110mm}{testPassed}{Результаты тестирования}{test}

Таким образом, все тесты были пройдены, и программа на ускорителе работает
верно.
